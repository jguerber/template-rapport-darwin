% !TeX root = main.tex

\documentclass[12pt]{article}

\usepackage[french]{babel}
\usepackage[T1]{fontenc}

\usepackage{lipsum}

% set page geometry
\usepackage{geometry}

\geometry{
    a4paper,
    left=25mm,
    right=25mm,
    top=25mm,
    bottom=25mm
}

% times new roman
\usepackage{mathptmx}

%linespacing
\usepackage{setspace}
\onehalfspacing{}

% useful packages
\usepackage{amsmath}
\usepackage{amsfonts}

\usepackage{graphicx}
\usepackage{wrapfig}
\usepackage{float}

\usepackage{framed}

\usepackage{multicol}

\usepackage{ifthen}
\usepackage{tikz}


%load bibliography
\usepackage[backend=biber, 
    citestyle=authoryear, 
    bibstyle=verbose, 
    datamodel=datamodels, % charge datamodels.dbx
    maxbibnames=10, 
    maxcitenames=2, 
    natbib, 
    sorting = nyt, 
    doi=true, 
    date = year]{biblatex}
\usepackage{csquotes}

%%%% Gestion de la biblio

% chargement
\addbibresource{biblio.bib}

% package pour refaire les macros
\usepackage{xpatch}

% asterisque devant options={extsym={*}}
\makeatletter
\providecommand{\bib@extsym}{}
\DeclareEntryOption[string]{extsym}{\renewcommand{\bib@extsym}{#1}}
\renewbibmacro*{begentry}{\textbf{\printtext{\bib@extsym}}}
\makeatother

% on enleve les champs useless
\renewbibmacro{in:}{} % "in" devant le nom de revue
\renewbibmacro{urldate}{} % date de téléchargement du pdf
\renewbibmacro{url}{} % url (redondant si on a le doi)

\AtEveryBibitem{
    \clearfield{issn} % issn
    \clearlist{language} % langue de l'article
}

% (pas compris pourquoi des fois ça doit etre renewbibmacro
% et des fois AtEveryBibItem mais ça marche)

% on ajoute mynote à la fin
\xapptobibmacro{finentry}{ \textbf{\printfield{mynote}}}{}{}

% citations en police normale
\AtBeginDocument{\protected\def\mkbibnamefamily#1{%
\textnohyphenation{#1}}}

% premier.e + (n-1) auteurices
\DefineBibliographyStrings{french}{andothers={
    \ifthenelse{\arabic{author} < 10}
    {et~al.}
    {et~\pgfmathparse{int(\arabic{author} - 1)}\pgfmathresult{} co-auteur.e.s}
}}

% définit \beginsupplement pour le début de l'annexe
\newcommand{\beginsupplement}{
        \setcounter{table}{0}
        \renewcommand{\thetable}{S\arabic{table}}%
        \setcounter{figure}{0}
        \renewcommand{\thefigure}{S\arabic{figure}}%
     }

% une commande qui crée une section non numérotée mais
% l'ajoute à la table des matières quand même
\newcommand{\newsection}[2]{
    \section*{#1} % sans le numéro
    \addcontentsline{toc}{section}{#2} % dans la TOC
}

% pareil pour les subsection
\newcommand{\newsubsection}[2]{
    \subsection*{#1}
    \addcontentsline{toc}{subsection}{#2}
}

\usepackage{xcolor}
\usepackage{titlesec}

% definitions de couleur
\definecolor{reddishBrown}{HTML}{683309}
\definecolor{brownishYellow}{HTML}{b47c4b}

% format des titres avec ou sans couleur custom
\titleformat{\section}
  {\LARGE\bfseries\color{reddishBrown}}{}{0em}{}[{\titlerule[0.8pt]}\vspace{0.5em}]

\titleformat{\subsection}
    {\Large\bfseries\color{brownishYellow}}{}{1em}{}

\titleformat{\subsubsection}
    {\bfseries}{}{1em}{}

% je pense que c'est ici qu'on pourrait rajouter des options pour la police des citations, la forme des références, la légende des figures, ...
 % inclure la feuille de style

\begin{document}

\pagenumbering{roman} % petits chiffres romains pour les pages hors 30 pages

\setcounter{tocdepth}{2}
\tableofcontents

\pagebreak

\section*{Résumé} %sans le numéro
\addcontentsline{toc}{section}{Résumé} %mais quand meme dans la toc

\lipsum[1]

\lipsum[2]

\newsection{Test de newsection}{Test}

\pagebreak
\pagenumbering{arabic} % chiffres arabes pour les 30 pages

\section*{Introduction}
\addcontentsline{toc}{section}{Introduction}


\subsection*{C'est une bonne situation ça, scribe?}
\addcontentsline{toc}{subsection}{Scribe}

Mais, vous savez, moi je ne crois pas qu'il y ait de bonne ou de mauvaise situation. Moi, si je devais résumer ma vie aujourd'hui avec vous, je dirais que c'est d'abord des rencontres, des gens qui m'ont tendu la main, peut-être à un moment où je ne pouvais pas, où j'étais seul chez moi \citep{pecresse_article_2022}. Et c'est assez curieux de se dire que les hasards, les rencontres forgent une destinée… Parce que quand on a le goût de la chose, quand on a le goût de la chose bien faite, le beau geste, parfois on ne trouve pas l'interlocuteur en face, je dirais, le miroir qui vous aide à avancer. Alors ce n'est pas mon cas, comme je le disais là, puisque moi au contraire, j'ai pu ; et je dis merci à la vie, je lui dis merci, je chante la vie, je danse la vie… Je ne suis qu'amour \citep{hetherington_two-_1975}! Et finalement, quand beaucoup de gens aujourd'hui me disent « Mais comment fais-tu pour avoir cette humanité ? », eh ben je leur réponds très simplement, je leur dis que c'est ce goût de l'amour, ce goût donc qui m'a poussé aujourd'hui à entreprendre une construction mécanique, mais demain, qui sait, peut-être seulement à me mettre au service de la communauté, à faire le don, le don de soi…

\pagebreak

Une autre page pour vérifier si la numérotation marche

\newsubsection{Deuxième subsection}{coucou}

\pagebreak

%small spacing for biblio
\singlespacing{}

\addcontentsline{toc}{section}{Références}
\printbibliography{}

\pagebreak
\pagenumbering{Roman} % grands chiffres romains pour l'annexe

%normal spacing for supplementary
\onehalfspacing{}
\beginsupplement{}
\section*{Annexe 1 : pourquoi la vie}
\addcontentsline{toc}{section}{Annexe sur la vie}

rt si c triste

\lipsum[2]

\end{document}